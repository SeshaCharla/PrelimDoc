\newcommand{\ZZatinformation}{}
\graphicspath{{graphics/}}
\makeatletter
  \def\input@path{{packages}}
\makeatother

\ConfigureBibliography

\usepackage{bm}
\usepackage{bold-extra}
\usepackage{fancyvrb}
  \DefineShortVerb{\|}
\usepackage{filehook}
\usepackage{fontspec}
\usepackage{listings}
\usepackage{cancel}
\usepackage{etoc}

% Include XMP data in a PDF document.
% See
%   http://mirrors.ibiblio.org/CTAN/macros/latex/contrib/hyperxmp/hyperxmp.pdf
% for more information including all the possible fields that can be set.
\usepackage{hyperxmp}
  \hypersetup{
    pdfauthor    = {Mark Senn},
    pdfcopyright = {Copyright \copyright\ 2024 by Mark Senn.  All rights reserved.},
    % Use `yyyy-mm' format where `yyyy' is year and `mm' is month.
    pdfdate      = {2025-05},
    pdfkeywords  = {LaTeX; Purdue University; PurdueThesis},
    pdflang      = {en},
    pdfpublisher = {Purdue University},
    pdfsubject   = {%
                     PurdueThesis is a LaTeX document class used for
                     master's bypass reports,
                     master's theses,
                     PhD dissertations,
                     and PhD preliminary reports.
                     This template demonstrates how to use PurdueThesis.%
                   },
    pdftitle     = {This is the Title},
  }

\usepackage{luacode}
\usepackage{makeidx}
  \makeindex
  \renewcommand{\indexname}{INDEX}
\usepackage{mathtools}
\usepackage{media9}
\usepackage{multicol}
\usepackage{pa-ditto}
\usepackage{pa-figure-dash}
\usepackage{pa-logos}
\usepackage{pdfpages}
\def\pa{\rotatebox[origin=c]{14}{\partial}}
\def\Fourier{\mathcal{F}}
\def\Laplace{\mathcal{L}}
\usepackage{pa-repeat}
\usepackage{placeins}
\let\st\relax
\usepackage{soul}
\usepackage{textcomp}
\usepackage{tikz}
  \usetikzlibrary{arrows.meta}
  \usepackage{circuitikz}
  \usepackage{menukeys}
  \usetikzlibrary{3d}
  \usetikzlibrary{calc,shadows,shapes.misc,shapes.symbols}
  \usetikzlibrary{decorations.text}
  \usetikzlibrary{decorations.pathmorphing} % noisy shapes
  \usetikzlibrary{fit}
  \usetikzlibrary{backgrounds}	% drawing the background after the foreground
\newcommand{\tabularspace}{\noalign{\vspace*{2pt}}}
\usepackage{xfrac}
\newcommand{\I}[1]{\MyRepeat{\indent}{#1}}
\long\def\MyI#1%
  {%
    {%
      \fontsize{8}{10}\tt
      \VerbatimInput
        [
          firstnumber = 1,
          numbers     = left,
          xleftmargin = 0.33in,
        ]%
        {#1}
    }%
  }
\newcommand{\MyIO}
  {%
    \input{z.out}

    {%
      \fontsize{8}{10}\tt
      \VerbatimInput
        [
          firstnumber = 1,
          numbers     = left,
          xleftmargin = 0.33in,
        ]
        {z.out}
    }
    \FloatBarrier
  }
\newcommand{\NL}{\mbox{}\\}
\newenvironment{inlinetable}
  {%
    \begingroup
      \singlespace
      \mbox{}\\[-9pt]%
      \noindent
      \hspace*{2\parindent}%
      \ignorespaces
  }
  {%
      \mbox{}\\
    \endgroup
  }
\listfiles
\usepackage{pa-typographic-conventions}
\usepackage{covington}
\usepackage{slgloss}
