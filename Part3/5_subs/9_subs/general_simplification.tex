\subsection{Some general simplification}
\begin{enumerate}
        \item Sum of two rate constants:
        \begin{align}
                k_{oxi} + k_{des} &= k_{od} = (m_{oxi} + m_{des}) T + (c_{oxi} + c_{des}) = m_{od} T + c_{od} \label{eqn::k_sum}
        \end{align}
        %===
        \item Product of residence time and rate constant:
        \begin{align*}
        k_i \tau  = \lr{m_i T + c_i} \frac{\tau_0}{F} = \bm{\frac{T}{F} & \frac{1}{F}} \bm{\tau_0 m_i \\ \tau_0 c_i}
        % \lr{\tau_0 - \tau_T T - \tau_F F}
        %           &= - \tau_T m_i T^2 - \tau_F m_i TF + \lr{m_i \tau_0 - \tau_T c_i} T - \tau_F c_i F + \tau_0 c_i \\
        %           &= \bm{-T^2 & - TF & T & -F & 1} \bm{\tau_T m_i \\
        %                                                \tau_F m_i \\
        %                                                \lr{m_i \tau_0 - \tau_T c_i} \\
        %                                                \tau_F c_i \\
        %                                                c_i}
        \end{align*}
        %====
        Let,
        \begin{align}
                \pmb \phi_\tau &= \bm{\frac{T}{F} & \frac{1}{F}}^T  \label{eqn::phi_tau}\\
                \pmb \theta_i &= \bm{\tau_0 m_i &
                                     \tau_0 c_i}^T          \label{eqn::theta_i}\\
                \therefore \: k_i \tau &= \pmb \phi_\tau^T \pmb \theta_i   \label{eqn::tau_ki}
        \end{align}
        %===
        \item Product of urea-dosing dynamics with residence time and rate constant:
        \begin{align*}
        k_{i} \tau \lr{\frac{\nu_u u_{inj}}{F^2}} &= \frac{u_{inj}}{F^2} \pmb \phi_\tau^T \pmb \theta_i \nu_u
        \end{align*}
        %==
        \begin{align}
        \pmb\phi_{\tau,ur}^T &= \frac{u_{inj}}{F^2}  \pmb \phi_\tau \\
        \pmb \theta_{i,ur}  &= \nu_u \theta_i
        \end{align}
\end{enumerate}
Thus,
\begin{align}
        k_i \tau \con{NH_3}^{in} &= \pmb \phi_{\tau, ur}^T \pmb \theta_{i, ur}   \label{eqn::k_tau_urea}
\end{align}
