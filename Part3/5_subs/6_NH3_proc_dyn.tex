\newpage
\section{Gaseous Ammonia Process Dynamics}
The gaseous ammonia process dynamics involve the adsorption and desorption dynamics. The gaseous ammonia that enters the
catalyst chamber gets adsorbed on the free sites on the catalyst surface at a rate proportional to the volumetric
concentration of the gaseous ammonia and the surface concentration of the free sites. A part of adsorbed ammonia then
desorbs back to the gas phase at a rate that is proportional to the surface concentrations of the adsorbed ammonia.
\begin{align*}
    NH_3 + \Theta_{free} &\xrightleftharpoons[k_{des}]{k_{ads}} NH_3^{ads}
\end{align*}

Thus, the rate of change of gaseous $NH_3$ on the catalyst surface can be modeled as:
\begin{align*}
    \frac{d \con{NH_3}^{scr}}{dt} &= k_{s2v} (-r_{ads} + r_{des}) \\
    r_{ads} &= k_{ads} \con{NH_3}^{in} \lr{\Gamma - \con{NH_3}^{ads}}\\
    r_{des} &= k_{des} \con{NH_3}^{ads}\\
    \dot{\con{NH_3}}^{scr} &= -k_{s2v}k_{ads} \con{NH_3}^{in} \lr{\Gamma - \con{NH_3}^{ads}} + k_{s2v} k_{des} \con{NH_3}^{ads}\\
\end{align*}

% ==============================================================================
\subsection{Molar conservation at the scale of residence time}

Similar to the ammonia adsorption/desorption process, at the end of every residence time, a fresh parcel of gaseous
reactants enters the catalyst chamber. Thus, the molar conservation for the gaseous $NH_3$ adsorption/desorption process
can be correlated with the inlet and outlet molar concentrations of the $NH_3$ at the beginning and the end of residence
time. Similar to the $NO_x$ process dynamics the measurement of $NH_3$ concentration at the outlet depends only on the
measurement of the $NH_3$ concentration at the inlet one residence time before. That is, there is no integrating effect
of the $NH_3$ within the sample time either. Writing in terms of volumetric concentrations, we have:

\begin{align*}
    \con{NH_3}^{out} (k + n\tau) &= \con{NH_3}^{in} (k + (n-1)\tau) + \int_{0}^{\tau}  \dot{\con{NH_3}}^{scr} (k + (n-1)\tau) dt
\end{align*}

Introducing the following two approximations:
\begin{enumerate}
    \item Zero-order-hold for the inlet concentration of $NH_3$:
        \begin{align*}
            \con{NH_3}^{in} (k + i \tau) &\approx \con{NH_3}^{in} (k) \qquad \forall i < n
        \end{align*}
    \item Using average surface concentration at the sample for the surface concentration of the adsorbed ammonia:
        \begin{align*}
            \con{NH_3}^{ads} (k + i \tau) &\approx \sigma(k) \qquad \forall i < n
        \end{align*}
\end{enumerate}

Thus,
\begin{align*}
    \dot{\con{NH_3}}^{scr} (k + i \tau) &= -k_{s2v}k_{ads} \con{NH_3}^{in} \lr{\Gamma - \sigma(k)} + k_{s2v} k_{des} \sigma(k)
    \qquad \forall i < n
\end{align*}


Thus, we have the following expression for the $NH_3$ process dynamics:

\begin{align*}
    \con{NH_3}^{out} (k + n\tau) &= \con{NH_3}^{in} (k) - \tau k_{s2v}k_{ads} \con{NH_3}^{in}(k) \lr{\Gamma - \sigma(k)} + \tau k_{s2v} k_{des} \sigma(k)\\
    \con{NH_3}^{out} (k + 1) &= \con{NH_3}^{in}(k) \lr{1 - \tau k_{s2v}k_{ads} \lr{\Gamma - \sigma(k)}} + \tau k_{s2v} k_{des} \sigma(k)
\end{align*}
