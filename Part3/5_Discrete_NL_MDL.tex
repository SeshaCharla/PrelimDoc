\chapter{Discrete Nonlinear Model Development for Urea SCR-ASC Dynamics}

One of the fundamental assumptions in the diesel engine SCR-ASC modelling, CSTR, reverts the causality in the reaction
rate constants, when the model order is reduced by considering a single cell. The present work circumvents the problem
by discarding the CSTR assumption and modelling the time evolution of the sensor signals when a "plug" or "parcel" of
the exhaust gasses flows through the chamber.

Such a time evolution introduces constraints on the model due to sampling limitations. To capture the transient
dynamics, the sampling time should be significantly smaller than the "residence time" of the reactants inside the
SCR-ASC chamber. If that is not the case as is the situation with the present available test and truck data, time integrated states assuming zero-order holds during the transients need to be introduced into the model and the input-output model can be derived from the resulting state-space model.

The modelling approach involves following the evolution of the measurement signals at the input and the output of the
system. As the plug of fluid flows through the chamber, these measurements can be correlated based on the conservation
of moles within the fluid plug.


\input{Part3/5_subs/4_NH3ads_proc_dyn.tex}
%\input{Part3/5_subs/5_NOx_proc_dyn.tex}
%\input{Part3/5_subs/6_NH3_proc_dyn.tex}
%\input{Part3/5_subs/7_urea_dosing_proc_dyn.tex}
%\input{Part3/5_subs/9_proc_dyn_mdl.tex}
%\input{Part3/5_subs/10_r_nox_mdl.tex}
%\input{Part3/5_subs/11_validation_results.tex}
%\input{Part3/5_subs/12_Conclusion.tex}
