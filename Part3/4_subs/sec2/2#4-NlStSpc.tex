\subsection{Nonlinear Statespace Model}
We have the mass (moles of reactants in/out) balance from the CSTR assumption:
\begin{multline} \label{eqn::mass_balance}
    %===
    \bm{\dot C_{NO} \\  \dot C_{NH_3} \\ \dot M_{NH_3} } = b_v
    \bm{
        -R_1 - F C_{NO}\\
        -R_{4F} + R_{4R} - F C_{NH_3}\\
        R_{4F} - R_{4R} - R_1 - R_3
    }
    +
    f_vb_v \bm{1 & 0 \\ 0 & 1 \\ 0 & 0} \bm{ C_{NO, in} \\ C_{NH_3, in}}
    %===
\end{multline}
Let,
\begin{align*}
    \bm{x_1 \\ x_2 \\ x_3 \\ x_4} = \bm{C_{NO} \\ C_{NH_3} \\ M_{NH_3} \\ C_{NH_3, in}} \qquad &
    \bm{u_1 \\ u_2 } = \bm{C_{NO, in} \\ u_{inj}}
\end{align*}
We have the nonlinear statespace model as:
\begin{multline}
    \label{eqn::full_nonlinear}
     \bm{\dot x_1 \\
        \dot x_2\\
        \dot x_3\\
        \dot x_4} =
    \bm{
        -f_{13} x_1 x_3
        -g_1 x_1
        \\
        %===
        -g_2 x_2
        + f_{23} x_2 x_3
        + g_{23} x_3
        + f_{24} x_4
        \\
        %===
        -f_{32} x_2 x_3
        -g_3 x_3
        -f_{31} x_1 x_3
        + g_{32} x_2
        \\
        %===
        -g_4 x_4
    }
    + \bm{b_{11} & 0\\
          0     & 0\\
          0     & 0\\
          0     & b_{42}  }\bm{u_1 \\ u_2 }
\end{multline}
The lumped coefficients of the states are defined in the appendix of the chapter.

The sensor used for $NO_x$ measurement is sensitive to the ammonia in the
tailpipe. This cross-sensitivity $(\chi)$ can be incorporated into the model by writing
the output equation as:
\begin{align}\label{eqn::ctrl_out}
    y_1 &= x_1 + \chi x_2
\end{align}


\itbf{Note:} The above nonlinear state-space model is not parsimonious and identifiable with the available measurements. One way to get an identifiable model is to linearize the above nonlinear model and come up with lumped model parameters for I/O equations based on the available measurements. The remainder of this chapter focuses on this particular approach and the resulting aging signatures.
