\subsection{Input-to-State Transfer Function Model}
Since we have uncorrupted test-cell data for $NO_x$ and $NH_3$ and all the
inputs at any given time, part of the state-to-input transfer function matrix
can be estimated using this data. Transforming state-space model from
section-\ref{eqn::lin_model} into frequency domain, the model structure would
be:
\begin{align*}
    \bm{\delta x_1 \\ \delta x_2 \\ \hline \delta  x_3 \\ \delta x_4} &= \bm{G_{11} & G_{12} & G_{13} & G_{14}\\
                G_{21} & G_{22} & G_{23} & G_{24}\\
                \hline
                G_{31} & G_{32} & G_{33} & G_{34}\\
                G_{41} & G_{42} & G_{43} & G_{44}\\
    }
    \bm{\delta u_1 \\ \delta u_2 \\ \delta T \\ \delta f_v}
\end{align*}

With the given data, having $NO_x$ and $NH_3$ concentration measurements along
with all the inputs, $G_{1i}'s, G_{2i}'s$ can be estimated. Thus, the transfer function matrix that can be
identified using the available data is:
\begin{align*}
    G &= \bm{G_{11} & G_{12} & G_{13} & G_{14}\\
                G_{21} & G_{22} & G_{23} & G_{24}}
\end{align*}

Ignoring the actual output equation and using $C$ matrix that considers that the
first two states are measured, the input-to-state transfer function matrix is
given by:
\begin{align*}
    G &= C(sI-A)^{-1} B= \frac{1}{\norm{sI-A}} M B\\
    \text{Let, } M &= C \lr{\adj\lr{sI-A}} = \adj\lr{sI-A}[1:2, :]
\end{align*}
Thus only the first two rows of the adjoint of the $(sI-A)$ matrix are required
for computing the transfer function matrix.
Rewriting G as:
\begin{align*}
    G &= \frac{1}{D} \bm{N_{11} & N_{12} & N_{13} & N_{14}\\
                         N_{21} & N_{22} & N_{23} & N_{24}}
\end{align*}


% \subsubsection{Denominator Polynomial}
We have:
\begin{align*}
    D = |sI-A| &= \lr{g_4 + s} \times \underbrace{\lr{s^3 + d_2 s^2 + d_1 s + d_0}}_{D^{(3)}}
\end{align*}
Where,
\begin{align*}
    d_0 &= \left(\bar f_{13} - \bar f_{23}\right) \bar x_{3} + \bar f_{31} \bar x_{1} + \bar f_{32} \bar x_{2} + \bar g_{2} + \bar g_{3} + \bar g_{1}\\
    &= \left(\bar k_{4F} \bar x_{2} + \bar k_{4F}\right) \Theta + \left(\bar k_{1} - \bar k_{4F}\right) \bar x_{3} \\
    &\quad + \frac{V^{2} \bar k_{1} \bar x_{1} + V \bar k_{3} + V k_{4R} + 2 \bar f_v}{V}
\end{align*}

$d_1$ and $d_2$ are quadratic polynomials of nominal state values and
the storage capacity.

% \begin{align*}
%     d_2 &=- \bar f_{13} \bar f_{23} \bar x_{3}^{2} + \left(\bar f_{13} \bar f_{32} \bar x_{2} + \bar f_{13} \bar g_{2} + \bar f_{13} \bar g_{3} - \bar f_{23} \bar f_{31} \bar x_{1} - \bar f_{23} \bar g_{3} - \bar f_{23} \bar g_{1} + \bar f_{32} \bar g_{23}\right) \bar x_{3}\\
%     &\qquad  -  \bar f_{23} \bar g_{32} \bar x_{2} + \bar f_{31} \bar g_{2} \bar x_{1} + \bar f_{31} \bar g_{1} \bar x_{1} + \bar f_{32} \bar g_{2} \bar x_{2} + \bar f_{32} \bar g_{1} \bar x_{2} + \bar g_{2} \bar g_{3} + \bar g_{2} \bar g_{1} + \bar g_{3} \bar g_{1} - \bar g_{23} \bar g_{32}\\
%     &= \bar k_{4F}^{2} \bar x_{2} \Theta^{2} + \frac{V \bar k_{1} \bar k_{4F} \bar x_{2} + V \bar k_{1} \bar k_{4F} + \bar k_{4F} k_{4R}}{V} \Theta\bar x_{3} + \frac{V^{2} \bar k_{1} \bar k_{4F} \bar x_{1} - V^{2} \bar k_{4F}^{2} \bar x_{2} + V \bar k_{3} \bar k_{4F} + 2 \bar F \bar k_{4F} \bar x_{2} + \bar F \bar k_{4F}}{V} \Theta \\
%     &\qquad -  \bar k_{1} \bar k_{4F} \bar x_{3}^{2} + \frac{- V^{2} \bar k_{1} \bar k_{4F} \bar x_{1} + V \bar k_{1} \bar k_{3} + V \bar k_{1} k_{4R} - V \bar k_{3} \bar k_{4F} - V \bar k_{4F} k_{4R} + \bar F \bar k_{1} - \bar F \bar k_{4F}}{V} \bar x_{3} \\
%     &\qquad \qquad + \frac{2 V^{2} \bar F \bar k_{1} \bar x_{1} + 2 V \bar F \bar k_{3} + 2 V \bar F k_{4R} + \bar F^{2}}{V^{2}}
% \end{align*}
% \begin{align*}
%     d_3 &= \left(- \bar f_{13} \bar f_{23} \bar g_{3} + \bar f_{13} \bar f_{32} \bar g_{23}\right) \bar x_{3}^{2} \\
%     &\qquad + \left(- \bar f_{13} \bar f_{23} \bar g_{32} \bar x_{2} + \bar f_{13} \bar f_{32} \bar g_{2} \bar x_{2} + \bar f_{13} \bar g_{2} \bar g_{3} - \bar f_{13} \bar g_{23} \bar g_{32} - \bar f_{23} \bar f_{31} \bar g_{1} \bar x_{1} - \bar f_{23} \bar g_{3} \bar g_{1} + \bar f_{32} \bar g_{1} \bar g_{23}\right) \bar x_{3} \\
%     &\qquad \qquad -  \bar f_{23} \bar g_{1} \bar g_{32} \bar x_{2} + \bar f_{31} \bar g_{2} \bar g_{1} \bar x_{1} + \bar f_{32} \bar g_{2} \bar g_{1} \bar x_{2} + \bar g_{2} \bar g_{3} \bar g_{1} - \bar g_{1} \bar g_{23} \bar g_{32}\\
%     &=  \bar k_{1} \bar k_{4F}^{2} \bar x_{2} \Theta^{2}\bar x_{3} + \frac{\bar F \bar k_{4F}^{2} \bar x_{2}}{V} \Theta^{2} + \frac{\bar k_{1} \bar k_{4F} k_{4R}}{V} \Theta\bar x_{3}^{2} + \frac{- V^{3} \bar k_{1} \bar k_{4F}^{2} \bar x_{2} + V^{2} \bar k_{1} \bar k_{3} \bar k_{4F} + V \bar F \bar k_{1} \bar k_{4F} \bar x_{2} + \bar F \bar k_{4F} k_{4R}}{V^{2}} \Theta\bar x_{3} \\
%     & \qquad + \frac{V^{2} \bar F \bar k_{1} \bar k_{4F} \bar x_{1} - V^{2} \bar F \bar k_{4F}^{2} \bar x_{2} + V \bar F \bar k_{3} \bar k_{4F} + \bar F^{2} \bar k_{4F} \bar x_{2}}{V^{2}} \Theta + \left(- \bar k_{1} \bar k_{3} \bar k_{4F} - \bar k_{1} \bar k_{4F} k_{4R}\right) \bar x_{3}^{2} \\
%     &\qquad \qquad + \frac{- V \bar F \bar k_{1} \bar k_{4F} \bar x_{1} + \bar F \bar k_{1} \bar k_{3} + \bar F \bar k_{1} k_{4R} - \bar F \bar k_{3} \bar k_{4F} - \bar F \bar k_{4F} k_{4R}}{V} \bar x_{3} + \frac{V \bar F^{2} \bar k_{1} \bar x_{1} + \bar F^{2} \bar k_{3} + \bar F^{2} k_{4R}}{V^{2}}
% \end{align*}

% \subsubsection{Numerator Matrix}
We have the numerator matrix of the transfer function matrix:
\begin{align*}
    N &= \bm{N_1 \\ N_2} = \bm{N_{11} & N_{12} & N_{13} & N_{14}\\
             N_{21} & N_{22} & N_{23} & N_{24}} = M B
\end{align*}

Where, $M$ is the first two rows of the adjoint matrix of $sI-A$. Calculating
the expressions for each of the elements of $N$ matrix symbolically:

\begin{minipage}{0.49\textwidth}
 \begin{align*}
    N_{11} &= (s + g_4)(n_{11_2} s^2 + n_{11_1} s + n_{11_0})\\
    %===
    N_{12} &= \bar b_{42} \bar f_{13} \bar f_{24} \bar x_{1} \left(\bar f_{32} \bar x_{3} - \bar g_{32}\right) \\
    &= \frac{\Theta \bar k_{1} \bar k_{4F} \bar x_{1} \omega_{u} b_{u} \left(- V + \bar x_{3}\right)}{V}\\
    %===
    N_{13} &= (s + g_4)(n_{13_2} s^2 + n_{13_1} s + n_{13_0})\\
    %===
    N_{14} &= n_{14_3} s^3 + n_{14_2} s^2 + n_{14_1} s + n_{14_0}\\
 \end{align*}
\end{minipage}
\begin{minipage}{0.49\textwidth}
 \begin{align*}
    %===
    N_{21} &= - \bar b_{11} \bar f_{31} \bar x_{3} \left(g_{4} + s\right) \left(\bar f_{23} \bar x_{2} + \bar g_{23}\right) \\
    %===
    N_{22} &= n_{22_2} s^2 + n_{22_1} s + n_{22_0}\\
    %===
    N_{23} &= (s + g_4)(n_{23_2} s^2 + n_{23_1} s + n_{23_0})\\
    %===
    N_{24} &= n_{14_3} s^3 + n_{14_2} s^2 + n_{14_1} s + n_{14_0}\\
\end{align*}
\end{minipage}

\itbf{Pole-Zero Cancellation:} Symbolic computations reveal that $s + g_4$ is
the common factor among all the numerator polynomials and the denominator. This
gets cancelled, removing all the direct effects of urea-dosing dynamics on
$\delta x_1, \delta x_2$ from temperature and $NO_x$ inputs. The flow rate
interacts with the urea-dosing dynamics as it affects the urea decomposition rate.

From previous calculations, we can arrive at a general structure for the
transfer function matrices. Let $^{(i)}$ denote the polynomial order of the
expression in $'s'$ (Laplace variable).

Thus,
\begin{align} \label{eqn::I2S_tf}
    G &= \bm{\frac{N_{11}^{(2)}}{D^{(3)}} &
             \frac{N_{12}^{(0)}}{\lrb{(s+g_4) D^{(3)}}} &
             \frac{N_{13}^{(2)}}{D^{(3)}} &
             \frac{N_{14}^{(3)}}{\lrb{(s+g_4)D^{(3)}}}
             \\
             \frac{N_{21}^{(0)}}{D^{(3)}} &
             \frac{N_{22}^{(2)}}{\lrb{(s+g_4)D^{(3)}}} &
             \frac{N_{23}^{(2)}}{D^{(3)}} &
             \frac{N_{24}^{(3)}}{\lrb{(s+g_4) D^{(3)}}}
    }
\end{align}

The above model structure information can be used to identify the transfer
function model for the linearized system. The transfer function model can be
converted into regression form and least-squares optimization can be used to
identify the model parameters. The regression form of the model would be:

\begin{align} \label{eqn::I2S_regression_form}
    \bm{\frac{d^4 x_1}{dt^4}\\
        \frac{d^4 x_2}{dt^4}}
    &= \bm{\pmb \phi_D(x_1)
            & \pmb \phi_{N_1}
            & \pmb 0\\
            \pmb \phi_D(x_2)
            & \pmb 0
            & \pmb \phi_{N_2}}
        \bm{\pmb \theta_D \\ \pmb \theta_{N_1} \\ \pmb \theta_{N_2}}
\end{align}

