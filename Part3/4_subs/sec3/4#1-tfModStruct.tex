\subsection{Transfer function model structure}

Since we have uncorrupted test-cell data for $NO_x$ and $NH_3$ and all the
inputs at any given time, part of the state-to-input transfer function matrix
can be estimated using this data. Transforming the state-space model from
section-\ref{eqn::lin_model} into frequency domain, the model structure would
be:
\begin{align*}
    \bm{\delta x_1 \\ \delta x_2 \\ \hline \delta  x_3 \\ \delta x_4} &= \bm{G_{11} & G_{12} & G_{13} & G_{14}\\
                G_{21} & G_{22} & G_{23} & G_{24}\\
                \hline
                G_{31} & G_{32} & G_{33} & G_{34}\\
                G_{41} & G_{42} & G_{43} & G_{44}\\
    }
    \bm{\delta u_1 \\ \delta u_2 \\ \delta T \\ \delta f_v}
\end{align*}

With the given data, $G_{1i}'s, G_{2i}'s$ can be estimated. The presumption of
this approach is that we should be able to get estimates of most of the required
parameters from these two rows. Thus, the transfer function matrix that can be
identified using the available data is:
\begin{align*}
    G &= \bm{G_{11} & G_{12} & G_{13} & G_{14}\\
                G_{21} & G_{22} & G_{23} & G_{24}}
\end{align*}

We have the linearized state-space system matrices from section-\ref{eqn::lin_model}:
\begin{align*}
    A &= \bm{-\lr{ \bar f_{13} \bar x_3 + \bar g_1 } &
                    0 &
                    -\bar f_{13} \bar x_1 &
                    0 \\
                    %===
                    0 &
                    \lr{ -\bar g_2 + \bar f_{23} \bar x_3 }&
                    \lr{\bar f_{23} \bar x_2 + \bar g_{23}}&
                    \bar f_{24} \\
                    %===
                    -\bar f_{31} \bar x_3 &
                    \lr{-\bar f_{32} \bar x_3 + \bar g_{32} } &
                    \lr{-\bar f_{32} \bar x_2 - \bar g_3 - \bar f_{31} \bar x_1 } &
                    0 \\
                    %===
                    0 & 0 & 0 & -g_4
                    }\\
    %==
    B &= \bm{ \bar b_{11} &
                            0 &
                            \delta f_{13} \bar x_1 \bar x_3 &
                            \lr{\delta b_{11} \bar u_1- \delta g_1 \bar x_1}
                            \\
                        %===
                        0&
                        0 &
                        \lr{-\delta g_{2_T} \bar x_2 + \delta f_{23} \bar x_2 \bar x_3 + \delta g_{23} \bar x_3} &
                        \lr{-\delta g_{2_F} \bar x_2 + \delta f_{24} \bar x_4 }
                        \\
                        %===
                        0&
                        0&
                        \lr{-\delta f_{32} \bar x_2 \bar x_3 - \delta g_3 \bar x_3 - \delta f_{31} \bar x_1 \bar x_3 + \delta g_{32} \bar x_2 } &
                        0
                        \\
                        %===
                        0&
                        \bar b_{42}&
                        0&
                        \delta b_{42} \bar u_2
                        }
\end{align*}
Ignoring the actual output equation from section-\ref{eqn::lin_model} and
considering that the first two states are measured, we get:
\begin{align*}
    C &= \bm{1 & 0 & 0 & 0 \\
             0 & 1 & 0 & 0}
\end{align*}
Thus the input-to-state transfer function matrix is given by:
\begin{align*}
    G &= C(sI-A)^{-1} B\\
      &= \lrb{\frac{1}{\norm{sI-A}}} C \lr{\adj\lr{sI-A}} B
      &&\lrb{\because (sI-A)^{-1} = \adj\lr{sI-A}/\det\lr{sI-A}}\\
      &= \frac{1}{\norm{sI-A}} \lr{\adj\lr{sI-A}[1:2, :]} B
      && \lrb{\because \text{Structure of C matrix}}\\
      &= \frac{1}{\norm{sI-A}} M B
      && \lrb{\because \text{Let, } M = C \lr{\adj\lr{sI-A}} = \adj\lr{sI-A}[1:2, :] }\\
\end{align*}
Thus only the first two rows of the adjoint of the $(sI-A)$ matrix are required
for computing the transfer function matrix.

Rewriting G as:
\begin{align*}
    G &= \frac{1}{D} \bm{N_{11} & N_{12} & N_{13} & N_{14}\\
                         N_{21} & N_{22} & N_{23} & N_{24}}
\end{align*}
