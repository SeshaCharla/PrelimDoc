Previous studies that employed rotor RPM feedback to enhance system dynamics, such as
\cite{pounds2009design},\cite{pounds2007system},\cite{mahony2012multirotor} relied on a second-order model around hover
conditions. This approach, while effective for minor input variations, may not be suitable for accommodating substantial
input changes. In contrast, \cite{franchi2017adaptive} implemented a hybrid controller using RPM feedback to linearize
actuator dynamics. However, this method necessitates low-level access to the Electronic Speed Controller's (ESC's)
microcontroller. A notable challenge when using commercial ESCs designed for manual control in autonomous operations is
their thrust-to-input curve is linearized for intuitive operability. Usually, this is achieved by introducing nonlinear
compensation in the speed controller of the ESC and the actual PWM input to the inverter controlling the BLDC motor
(Fig.-\ref{fig::bldc_diag}). However, attempting to estimate this compensation through system identification techniques
proved to be unfeasible.
%===
\begin{figure}[h]
    \centering
    \includegraphics[width = 0.7\textwidth]{Part2/figs/3_figs/esc_schematic.png}
    \caption{Schematic of ESC with BLDC motor}
    \label{fig::bldc_diag}
\end{figure}
%===

The present work introduces an adaptable input definition ($u_\omega$) capable of accommodating any nonlinear
compensatory mechanisms within the ESC that define ($u_p$). Utilizing this input definition, the nonlinear control model
of the system is developed. This model enables the design of improved controllers capable of explicitly mitigating
uncertainties, whose bounds are estimated experimentally.
