\section{Effect of propeller angular momentum}
\begin{figure}[h]
    \begin{minipage}{0.49\textwidth}
        \begin{figure}[H]
            \centering
            \includegraphics[width = 0.9\textwidth]{Part2/figs/1_figs/prop_momentum/hex_copt-1.jpg}
        \end{figure}
    \end{minipage}
        \begin{minipage}{0.49\textwidth}
        \begin{figure}[H]
            \centering
            \includegraphics[width = 0.8\textwidth]{Part2/figs/1_figs/prop_momentum/hex_copt-2.png}
        \end{figure}
    \end{minipage}
\end{figure}
%
\subsection{Propeller angular velocity vectors}
Let, $b$ be the body-frame with $\{\pmb{b_x \, b_y \, b_z}\}$ as 3 mutually perpendicular right-handed unit vectors shuch that $\pmb{b_z}$ points downwards (in the direction of gravity), $\pmb{b}_x$ point forward and $\pmb{b}_y$ such that they form the right-handed system. let, $\pmb{n}_i$ be the unit vectors pointing 'upwards' perpendicular to the rotor-plane of $i^{th}$ propeller and $\pmb{r}_i$ be the radial vectors pointing outwards along the $i^{th}$ propeller shaft. the propeller tilt-angle $\alpha_i$ is conisdered positive along $\pmb{r}_i$, i.e., radially outwards.
\begin{align*}
   \therefore \alpha_i &= (-1)^{i} \alpha \qquad (\alpha_1 = -\alpha, \; \alpha_2 = \alpha \hdots) \\
   \implies \cos \alpha_i &= \cos \alpha  = c_{\alpha} \quad and\\
   \sin \alpha_i &= (-1)^{i} \sin \alpha = (-1)^{i} s_{\alpha}
\end{align*}
%===
Let, $\gamma_i$ be the angle of the $i^{th}$ propeller shaft along $\pmb{b_z}$ (clockwise), i.e.,
\begin{figure}[H]
    \begin{minipage}{0.49 \textwidth}
            \begin{align*}
            \gamma_i &= \frac{i\pi}{3} - \frac{\pi}{6} = (2i-1)\frac{\pi}{6}\\
            \implies \pmb r_i &= \cos \gamma_i \pmb b_x + \sin \gamma_i \pmb b_y
            \end{align*}
    \end{minipage}
    \begin{minipage}{0.49 \textwidth}
        \begin{table}[H]
            \centering
            \begin{tabular}{c c c c }
                \hline \hline
                $i$ & $\gamma_i$ & $c_{\gamma_i}$ & $s_{\gamma_i}$ \\ \hline \hline
                $1$ & $\frac{\pi}{6} = 30^o$   & $\frac{\sqrt{3}}{2}$ & $\frac{1}{2}$ \\
                $2$ & $\frac{\pi}{2} = 90^o$   & $0$ & $1$ \\
                $3$ & $\frac{5\pi}{6} = 150^o$ & $-\frac{\sqrt{3}}{2}$ & $\frac{1}{2}$ \\
                $4$ & $\frac{7\pi}{6} = 210^o$ & $-\frac{\sqrt{3}}{2}$ & $-\frac{1}{2}$ \\
                $5$ & $\frac{3\pi}{2} = 270^o$ & 0 & $-1$ \\
                $6$ & $\frac{11\pi}{6} = 330^o$& $\frac{\sqrt{3}}{2}$ & $-\frac{1}{2}$ \\
                \hline \hline
            \end{tabular}
        \end{table}
    \end{minipage}
\end{figure}
%===
We have the propeller normals in body-frame:
\begin{align*}
    \pmb{n}_i &= -\cos \alpha_i \pmb{b}_z + \sin \alpha_i ( \pmb b_z \times \pmb r_i)\\
    &= -c_{\alpha} \pmb b_z + (-1)^{i} s_{\alpha}( \cos \gamma_i \pmb b_y - \sin \gamma_i \pmb b_x) \qquad
    [\because ( \pmb b_z \times \pmb r_i) = \pmb b_z \times (\cos \gamma_i \pmb b_x + \sin \gamma_i \pmb b_y)]\\
    %---
    &= -(-1)^i s_{\alpha} s_{\gamma_i} \pmb b_x + (-1)^{i} s_{\alpha} c_{\gamma_i} \pmb b_y - c_{\alpha} \pmb b_z
\end{align*}
$$\therefore \pmb{n}_i = \begin{bmatrix}
    \underbrace{(-1)^{i+1} s_{\alpha} s_{\gamma_i}}_{n_{ix}} &
    \underbrace{(-1)^{i} s_{\alpha} c_{\gamma_i}}_{n_{iy}} &
    \underbrace{- c_{\alpha}}_{n_{iz}}
\end{bmatrix} \begin{bmatrix}
    \pmb b_x \\ \pmb b_y \\ \pmb b_z
\end{bmatrix}
$$
Thus, using the above definitions, we have the angular velocity vecotor of $i^{th}$ propeller:
$${}^B\pmb \omega_i = (-1)^{i} \omega_i \pmb n_i \qquad \text{where, }\omega_i \geq 0 \quad \forall i = 1, 2, \hdots, 6$$
%===
% \subsection{Attitude and euler angles $(\phi, \theta, \psi)$}
% Let, $\{ \pmb a_x, \pmb a_y, \pmb a_z \}$ be three mutually perpendicular,
% right-handed vecotors in intertial frame $R$, such that they coinside with
% $\{ \pmb b_x, \pmb b_y, \pmb b_z\}$ respectively when $\pmb b_z$ is aligned to
% acceleration due to gravity $\pmb g$.
% Let, $\{ \pmb l_x, \pmb l_y, \pmb l_z\}$ be mutually perpendicular,
% right-handed directed line segments along $\{ \pmb a_x, \pmb a_y, \pmb a_z \}$
% initally.
%
% The 'attitude' of B relative to $\{ \pmb a_x, \pmb a_y, \pmb a_z \}$ can be
% specified interms of  three angles $\phi, \theta, \psi$ generated as follows:
% Performing successive right handed rotations of amount $\phi$ about $\pmb l_x$,
% $\theta$ about $\pmb l_y$ and $\psi$ about $\pmb l_z$ results in $\{ \pmb l_x,
% \pmb l_y, \pmb l_z\}$ coinside with $\{ \pmb b_x, \pmb b_y, \pmb b_z\}$.
%
% The above successive rotations can be represented as the following rotation matrics:
% \begin{align*}
%     R_{\phi} = \begin{bmatrix}
%         1 & 0 & 0\\
%         0 & c_{\phi} & s_{\phi}\\
%         0 & -s_{\phi} & c_{\phi}
%     \end{bmatrix} \qquad
%     R_{\theta} = \begin{bmatrix}
%         c_{\theta} & 0 & -s_{\theta}\\
%         0 & 1 & 0 \\
%         s_{\theta} & 0 & c_{\theta}
%     \end{bmatrix} \qquad
%     R_{\psi} = \begin{bmatrix}
%         c_{\psi} & s_{\psi} & 0\\
%         -s_{\psi} & c_{\psi} & 0\\
%         0 & 0 & 1
%     \end{bmatrix}
% \end{align*}
% $$\implies [\pmb b_x, \pmb b_y, \pmb b_z]^T = R_{\psi}R_{\theta}R_{\phi} [\pmb a_x, \pmb a_y, \pmb a_z]^T$$
% We have,
% $$R_{\psi}R_{\theta}R_{\phi} =\displaystyle \left[\begin{matrix}\cos{\left(\psi
% \right)} \cos{\left(\theta \right)} & \sin{\left(\phi \right)}
% \sin{\left(\theta \right)} \cos{\left(\psi \right)} + \sin{\left(\psi \right)}
% \cos{\left(\phi \right)} & \sin{\left(\phi \right)} \sin{\left(\psi \right)} -
% \sin{\left(\theta \right)} \cos{\left(\phi \right)} \cos{\left(\psi \right)}\\-
% \sin{\left(\psi \right)} \cos{\left(\theta \right)} & - \sin{\left(\phi
% \right)} \sin{\left(\psi \right)} \sin{\left(\theta \right)} + \cos{\left(\phi
% \right)} \cos{\left(\psi \right)} & \sin{\left(\phi \right)} \cos{\left(\psi
% \right)} + \sin{\left(\psi \right)} \sin{\left(\theta \right)} \cos{\left(\phi
% \right)}\\\sin{\left(\theta \right)} & - \sin{\left(\phi \right)}
% \cos{\left(\theta \right)} & \cos{\left(\phi \right)} \cos{\left(\theta
% \right)}\end{matrix}\right]$$
%===
% \subsubsection{Angular velocity vector of the body ($\Omega$)}
% %===
% We have the instantaneous angular velocity of the body as:
% \begin{align*}
%     {}^R\pmb \Omega^B &= \dot \phi \pmb l_x + \dot \theta \pmb l_y + \dot \psi \pmb l_z\\
% \end{align*}
% The vectors $\{ \pmb l_x, \pmb l_y, \pmb l_z\}$ can be written in terms of $\{ \pmb b_x, \pmb b_y, \pmb b_z\}$ as follows:
% \begin{align*}
%     \pmb l_z &= \pmb b_z \qquad [\because R_{\psi} \text{ is about } \pmb l_z]\\
%     %===
%     \pmb l_y &= R_{\psi}^T [2, :]\begin{bmatrix}
%     \pmb b_x \\ \pmb b_y \\ \pmb b_z\\
% \end{bmatrix} = s_{\psi} \pmb b_x + c_{\psi} \pmb b_y\\
%     %===
%     \pmb l_x &= [R_{\theta}^T R_{\psi}^T][1,:] \begin{bmatrix}
%     \pmb b_x \\ \pmb b_y \\ \pmb b_z\\
% \end{bmatrix} = \begin{bmatrix}
%     c_{\theta}c_{\psi} & -c_{\theta}s_{\psi} & s_{\theta}
% \end{bmatrix}\begin{bmatrix}
%     \pmb b_x \\ \pmb b_y \\ \pmb b_z\\
% \end{bmatrix}
% \end{align*}
% %====
% Hence,
% \begin{align*}
%     {}^R\pmb \Omega^B &=
%         \dot \phi \begin{bmatrix} c_{\theta}c_{\psi} & -c_{\theta}s_{\psi} &
%         s_{\theta} \end{bmatrix}\begin{bmatrix} \pmb b_x \\ \pmb b_y \\ \pmb
%         b_z\\\end{bmatrix} +
%         \dot \theta \begin{bmatrix} s_{\psi} & c_{\psi} & 0
%         \end{bmatrix}\begin{bmatrix} \pmb b_x \\ \pmb b_y \\ \pmb
%         b_z\\\end{bmatrix} +
%         \dot \psi \begin{bmatrix} 0 & 0 & 1\end{bmatrix}
%         \begin{bmatrix} \pmb b_x \\ \pmb b_y \\ \pmb
%         b_z\\\end{bmatrix}
% \end{align*}
% %===
% Let,
% \begin{align*}
%     {}^R\pmb \Omega^B  &= \Omega_x \pmb b_x + \Omega_y \pmb b_y + \Omega_z \pmb b_z\\
% \end{align*}
% %===
% combaring the coefficients,
% \begin{align*}
%     \begin{bmatrix}
%         \Omega_x \\ \Omega_y \\ \Omega_z
%     \end{bmatrix}
%     &=
%     \begin{bmatrix}
%         c_{\theta}c_{\psi} &  s_{\psi} & 0\\
%         -c_{\theta}s_{\psi} & c_{\psi} & 0\\
%         s_{\theta} & 0 & 1
%     \end{bmatrix}
%     \begin{bmatrix}
%         \dot \phi \\ \dot \theta \\ \dot \psi
%     \end{bmatrix}
% \end{align*}
% %===
% Thus, the instantaneous angular velocity is purely a function of $\dot \phi , \dot \theta , \dot \psi $.
%===
\subsection{Inertial moments due to propeller angular momentum}
Let, ${}^R \pmb \Omega^B$ be the angular velocity of the drone in inertial frame. We have, angular velocity of the propeller $i$ in inertial frame:
$${}^R \pmb \omega_i = {}^R \pmb \Omega^B + {}^B \pmb \omega_i$$
Hence, angular acceleration in reference frame:
\begin{align*}
    \frac{{}^R d \pmb \omega_i}{dt} &= \frac{{}^R d}{dt}\left( {}^R \pmb \Omega^B + {}^B \pmb \omega_i \right)
    = \underbrace{\frac{{}^R d \pmb \Omega^B}{dt}}_{{}^R\pmb\alpha^B} + \frac{{}^R d{}^B \pmb \omega_i}{dt}\\
    %==
\text{Consider,} \qquad & \\
%===
    \frac{{}^R d{}^B \pmb \omega_i}{dt} &= \frac{{}^R d}{dt} (-1)^{i} \omega_i \pmb n_i
    = \underbrace{(-1)^{i} \frac{{}^R d\omega_i }{dt} \pmb n_i}_{{}^B \pmb \alpha_i} +
    (-1)^{i} \omega_i \frac{{}^R d\pmb n_i}{dt} \\
\text{We have,} \qquad &\\
    \frac{{}^R d\pmb n_i}{dt} &= {}^R \pmb \Omega^B \times \pmb n_i
    = \begin{bmatrix}
        \Omega_y n_{iz} - \Omega_z n_{iy} &
        \Omega_z n_{ix} - \Omega_x n_{iz} &
        \Omega_x n_{iy} - \Omega_y n_{ix}
    \end{bmatrix}
    \begin{bmatrix}
    \pmb b_x \\ \pmb b_y \\ \pmb b_z\\
    \end{bmatrix}
\end{align*}
%
Let, $\dot \varepsilon^T  = \begin{bmatrix} \dot \phi & \dot \theta & \dot \psi \end{bmatrix}$. We have,
%
\begin{alignat*}{3}
        \Omega_x &=  c_{\theta}c_{\psi} \dot \phi + s_{\psi} \dot \theta \qquad
        \Omega_y &&= -c_{\theta}s_{\psi} \dot \phi + c_{\psi} \dot \theta \qquad
        \Omega_z &&= s_{\theta} \dot \phi + \dot \psi\\
        n_{ix}   &=  (-1)^{i+1} s_{\alpha} s_{\gamma_i} \qquad
        n_{iy}   &&= (-1)^{i} s_{\alpha} c_{\gamma_i}  \qquad \qquad
        n_{iz}   &&= -c_{\alpha}
\end{alignat*}
%
% Calculating the individual terms:
% \begin{align*}
%     \Omega_y n_{iz} - \Omega_z n_{iy} &=
%     \begin{bmatrix}
%         \dot \phi & \dot \theta & \dot \psi
%     \end{bmatrix}
%     \begin{bmatrix}
%         (-1)^{i+1} s_{\theta} s_{\alpha} c_{\gamma_i} + s_{\psi} c_{\theta} c_{\alpha}\\
%         -c_{\psi} c_{\alpha}\\
%         (-1)^{i+1} s_{\alpha} c{\gamma_i}
%     \end{bmatrix}\\
%     %===
%     \Omega_z n_{ix} - \Omega_x n_{iz} &=
%     \begin{bmatrix}
%         \dot \phi & \dot \theta & \dot \psi
%     \end{bmatrix}
%     \begin{bmatrix}
%         (-1)^{i+1} s_{\theta} s_{\alpha} s_{\gamma_i} + c_{\psi} c_{\theta} c_{\alpha}\\
%         s_{\psi} c_{\alpha}\\
%         (-1)^{i+1} s_{\alpha} s_{\gamma_i}\\
%     \end{bmatrix}\\
%     %===
%     \Omega_x n_{iy} - \Omega_y n_{ix} &=
%     \begin{bmatrix}
%         \dot \phi & \dot \theta & \dot \psi
%     \end{bmatrix}
%     \begin{bmatrix}
%         (-1)^{i} c_{\theta}s_{\alpha} (c_{\gamma_i}c_{\psi} - s_{\gamma_i} s_{\psi})\\
%         (-1)^{i} s_{\alpha} (s_{\gamma_i} c_{\psi} + c_{\gamma_i} s_{\psi})\\
%         0
%     \end{bmatrix}
% \end{align*}
Hence,
\begin{align*}
    \frac{{}^R d\pmb n_i}{dt} &= {}^R \pmb \Omega^B \times \pmb n_i\\
    &= \begin{bmatrix}
        \dot \phi & \dot \theta & \dot \psi
    \end{bmatrix}
    \underbrace{\begin{bmatrix}
        (-1)^{i+1} s_{\theta} s_{\alpha} c_{\gamma_i} + s_{\psi} c_{\theta} c_{\alpha}
        & (-1)^{i+1} s_{\theta} s_{\alpha} s_{\gamma_i} + c_{\psi} c_{\theta} c_{\alpha}
        & (-1)^{i} c_{\theta}s_{\alpha} c_{\gamma_i + \psi}
        \\
        -c_{\psi} c_{\alpha}
        & s_{\psi} c_{\alpha}
        & (-1)^{i} s_{\alpha} s_{\gamma_i + \psi}
        \\
        (-1)^{i+1} s_{\alpha} c{\gamma_i}
        & (-1)^{i+1} s_{\alpha} s_{\gamma_i}
        & 0
    \end{bmatrix}}_{G_i}
    \underbrace{\begin{bmatrix}
    \pmb b_x \\ \pmb b_y \\ \pmb b_z\\
    \end{bmatrix}}_{\pmb b}
\end{align*}
%
The inertial moment associated with the above motion is:
\begin{align*}
    M_{g_i} =  I_p \times (-1)^{i} \omega_i \left[\dot \varepsilon^T  G_i \pmb b \right]
\end{align*}
where, $I_p$ is the moment-of-inertia matrix of the propeller.
%
\subsubsection{Result of opposing spin for diagonally opposing propellers}
%
The opposing propeller pairs from the given configuration are $(1, 4), (2, 5) \text{ and } (3, 6)$. From the above derivation of $G_i$ matrix, it can be concluded that it is same for opposing propeller paris (the $-ve$ sign difference due to $(-1)^i$ or $(-1)^{i+1}$ is compensated by the negative sign difference in $c_{\gamma_i} \text{ and } s_{\gamma_i}$ for the opposing propellers), i.e.,
$$ G_{i} = G_{i+3} \qquad i = 1, 2, 3$$

Hence, the opposing propellers create inertial moments along the same line but in opposing directions. Hence the difference in the propeller velocities will cause the inertial moments in that particular direction. Thus we have, the inertial moments resulting from the propeller velocity difference:
%
\begin{align*}
    \delta M_{g_{14}} &=   I_p \delta \omega_{14} \left[\dot \varepsilon^T  G_{1 4} \pmb b \right] \qquad \delta \omega_{14} = \omega_4 - \omega_1, \quad G_{14} = G_1 = G_4\\
    %===
    \delta M_{g_{25}} &=   I_p \delta \omega_{25} \left[\dot \varepsilon^T  G_{2 5} \pmb b \right] \qquad \delta \omega_{25} = \omega_5 - \omega_2, \quad G_{25} = G_2 = G_5\\
    %==
    \delta M_{g_{36}} &=   I_p \delta \omega_{36} \left[\dot \varepsilon^T  G_{3 6} \pmb b \right] \qquad \delta \omega_{14} = \omega_6 - \omega_3, \quad G_{36} = G_3 = G_6\\
\end{align*}
%
Thus, when the tilted configuration is used under full actuation by introducing unbalanced propeller velocities, that introduces effects of unbalanced angular momentum that can only be compensated for using propeller RPM feedback. Further, such feedback can also be used for monitoring the performance of the actuator and use it for purely moment generation in the case of propeller breakage.
