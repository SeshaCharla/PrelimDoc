\section{RPM Measurement}
Real-time angular velocity measurement involves counting commutation signals
transmitted via an auxiliary cable from the ESC. This is achieved by utilizing
an interrupt trigger to record the time elapsed since the previous commutation
in an array, which resets after a defined sampling period. This approach
enhances measurement resolution compared to counting commutations within a
specific sampling period. Let $\hat \omega_{t_k}$ represent angular velocity
measured using commutation durations ($\delta t$) and $\hat \omega_{n_C}$
denote angular velocity measured using commutation counts ($\delta n_C$) within
a sampling period \cite{PX4-autopilot-rpm}. Commutation duration, denoted as
$\delta t$, is determined by a higher frequency counter, $f_t$, which is not
bound by the overall system sampling rate. Therefore, we can express $\delta t$
as $t_k / f_t$.

%===
\begin{align}
    \hat \omega_{n_C} &= \frac{2 \pi}{N_p n_C} f_s \, rad/s, \quad
    \hat \omega_{\delta t} = \frac{2 \pi}{N_p  \delta t}  = \frac{2 \pi f_t}{N_p t_k} \, rad/s
\end{align}
Where,
\begin{align*}
    N_p &- \text{No. of pole pairs in the motor (here, 7)}\\
    f_s &- \text{Sampling frequency}\\
    \delta t &- \text{Durtion between two commutations} = \frac{n_C}{f_s} = \frac{t_k}{f_t}\\
    n_C &- \text{No. of commutations within a sampling period}
\end{align*}
%===
In both cases, the minimum angular velocity from which the measurement begins is $\omega_{min} = \frac{2 \pi f_s}{N_p} = \frac{2 \pi f_t}{N_p t_{k_{max}}}$. This is because no commutation would be registered at a speed slower than $\omega_{min}$.

As for the resolution of our measurement processes:
%===
\begin{align}
    \min{\abs{\delta \omega_{n_C}}} &= \frac{2 \pi \min{\delta n_C}}{N_p n_C^2} f_s = \frac{2 \pi }{N_p n_C^2} f_s \\
    \min{\abs{\delta \omega_{t_k}}} &= \frac{2 \pi \min{\delta t_k}}{N_p t_k^2} f_t = \frac{2 \pi}{N_p t_k^2} f_t\\
    f_t >> f_s & \implies  t_k >> n_C \\
    \implies & \min{\abs{\delta \omega_{t_k}}} < \min{\abs{\delta \omega_{n_C}}}
\end{align}
%===
Thus higher resolution in rpm measurements can be achieved by choosing a arbitrarily high-frequency clounter.

Given that multiple measurements of $\delta t$ occur within a single sampling
instance and that the process is susceptible to noise due to interrupt skips,
we adopt a median-based approach to estimate angular velocity at the end of a
given sampling instance. Unlike other averaging methods, using the median
effectively eliminates spikes in $\delta t$ that can occur when an interrupt is
skipped. Detailed pseudocode for this algorithm is presented in
Algorithm-\ref{alg::ISR} and Algorithm-\ref{alg::get_rpm}.
%===

\begin{minipage}{0.49\textwidth}
\RestyleAlgo{boxruled}
\LinesNumbered
\begin{algorithm}[H]
    \footnotesize
    $n_C \peq 1$\;
    \Comment*[l]{$C_c$ - Counter value at current interrupt}
    \Comment*[l]{$P_c$ - Counter value at previous interrupt}
    $\delta t_k = C_c - P_c$\;
    \If{ $\delta t_k \leq 0$ }
        {\Comment*[l]{Correcting for integer overflow}
         $\delta t_k \peq 2^{32}$}
    $\pmb{\delta t_k} [n_C-1] =  \delta t_k$\;
    $P_c = C_c$
    \caption{\small Interrupt Service Routine}
    \label{alg::ISR}
\end{algorithm}
\end{minipage}
%===
\begin{minipage}{0.49\textwidth}
\RestyleAlgo{boxruled}
\LinesNumbered
\begin{algorithm}[H]
    \footnotesize
    $N_p = 7$\;
    $f_t = 10^6$\;
    $M = \frac{2 \pi}{N_p} \times f_t$\;
    \eIf {$n_C > 0$ and $n_C \leq n_{C_{max}}$ and $\norm{n_C - n_{C_{old}}} \leq \delta n_{C_{max}}$}
        {\Comment*[l]{Median removes spikes in the data}
         $\omega = \frac{M}{\text{Median}(\pmb{\delta t_k})}$\;
         $ \omega_{old} = \omega$\;
         $n_{C_{old}} = n_C$\;
        }
         {$\omega = \omega_{old}$}
    $n_C = 0$
    $\pmb{\delta t_k} = \pmb 0$\;
    \caption{\small RPM Measurement Function}
    \label{alg::get_rpm}
\end{algorithm}
\end{minipage}

The measurement algorithm's validation process is conducted by comparing its results against the ESC data-log, as depicted in Fig.-\ref{fig::valid}. These validation plots not only confirm the algorithm's accuracy but also reveal the presence of non-linear input compensation mechanisms within the ESC.
%===
\begin{figure}[H]
    \centering
    \includegraphics[width = 0.6\textwidth]{Part2/figs/2_figs/rpm_feedback/rpm_meas.eps}
    \caption{RPM vs PWM with and without propeller}
    \label{fig::valid}
\end{figure}
%===
