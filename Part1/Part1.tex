\chapter{Introduction}
The economical and safety related reasons had spurred ever-increasing interest in the theory and applications of
model-based fault detection and diagnosis methods. The use of parameter estimation techniques for fault detection of
mechanical systems was first demonstrated in \cite{baskiotis1979parameter}, \cite{geiger1982monitoring}
\cite{filbert1982quality}, \cite{hohmann1977automatic}. This as a part of larger class of fault detection based on
analytical redundancy. The development of process fault detection methods based on modelling, parameter and state
estimation was summarized by \cite{isermann1984process}, \cite{isermann1997trends}. Parity equation-based based methods
were treated early in \cite{chow1984analytical} and then later developed by Patton and others in \cite{patton1991review}
and \cite{hofling1994detection}.

Basic terminology related to fault diagnosis can be found in \cite{hofling1994detection} and \cite{ding2005model}. The following terms need the specific attention:

\begin{itemize}
        \item \itbf{Fault Detection}: Determination of faults present in the system at the time of detection.
        \item \itbf{Fault Isolation}: Determination of the kind, location and time of detection of a fault. Follows fault detection.
        \item \itbf{Fault Identification}: Determination of the size and time variant behavior of the fault. Follows fault isolation.
        \item \itbf{Fault Diagnosis}: Determination of the kind, size, location and time of detection of a fault. Follows fault detection. Includes fault detection and identification.
        \item \itbf{Residual}: A fault indicator signal, based on the deviation between the measurements and model equation based computations.
        \item \itbf{Analytical redundancy}: Use of more (not necessarily identical) ways to determine a variable, where
                one way uses a mathematical process model in analytical form. Analytical or functional redundancy
                exploits redundant analytical relationships among various measured variables of the monitored process \cite{chen2012robust}.
        \item \itbf{Fault Classification terminology}:
        \begin{itemize}
                \item \itbf{Time dependency of fault}:
                \begin{itemize}
                        \item \itbf{Abrupt Fault}: Fault modelled as stepwise function. It represents bias in the monitored signal.
                        \item \itbf{Incipient fault}: Fault modelled by using ramp signals. It represents drift of the monitored signal.
                        \item \itbf{Intermittent fault}: Combination of impulses with different amplitudes.
                \end{itemize}
                \item \itbf{Analytical type of fault}:
                \begin{itemize}
                        \item \itbf{Analytical redundancy}: Influences a variable by an addition of the fault itself. They may represent, e.g., offsets of sensors.
                        \item \itbf{Multiplicative fault}: Are represented by the product of a variable with the fault itself. They can appear as parameter changes within a process.
                \end{itemize}
        \end{itemize}
\end{itemize}


Consistency checking in analytical redundancy is normally achieved through a comparison between a measured signal with
estimated values. The estimation is generated by a mathematical model of the considered plant. The comparison is done
using the residual quantities which are computed as differences between the measured signals and the corresponding
signals generated by the mathematical model. Observers, Parity equations and Identification (Parameter Estimation) are
some basic model based residual generation methods. The characteristic quantities or features from fault detection
methods show stochastic behavior with mean values and variances. Deviations from the normal behavior must then be
detected by methods of change detection.

%=======================================================================================================================
\section{Change Detection}
\section{Model Parameter Estimation: Linear in parameters form}
\subsection{Relationship between RLS and Kalman Filter}
\section{Set Membership Identification}
\subsection{Relationship between SMI and Particle Filter}
\subsection{Applications to Fault Detection}
\subsection{Using SMI for setting parameter bounds in ARC}
\section{Desired Compensation Adaptive Robust Control}
